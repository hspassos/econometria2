Estimativa de Função de Produção

Seja a função de produção Cobb-Douglas (log-linear):

$y_{jt} = \beta_k k_{jt} + \beta_l l_{jt} + \omega_{jt}$ (1)

tal que $x = ln(X)$, X é uma variável aleatória, y é o ln do produto (medido como valor adicionado) da firma j no período t, k é o ln do estoque de capital e $\omega$ a produtividade total dos fatores (PTF).
O problema básico da estimação dos parâmetros da função de produção diz que as variáveis k e l podem ser correlacionadas com o termo de erro, se estimada por OLS. Além de que k e l podem ser escolhidos juntamente com y.
Neste problema, a função de produção (1) deve ser estimada usando o método de dois estágios de Ackerberg, Caves e Fraser (2015). O método é baseado em modelo estrutural de Olley e Pakes (1996) e discussões/avanços longos de Levinsohn e Petrin (2003) e Wooldridge (2009). O método é resumidamente apresentado abaixo – abstraindo da discussão teórica e da estratégia de identificação.

A estratégia de estimação é baseada em dois estágios. No primeiro estágio é computado um modelo semiparamétrico para identificar um termo de erro que ajuda na correta identificação dos parâmetros $\beta_k$ e $\beta_l$ em (1). No segundo estágio os parâmetros $\beta_k$ e $\beta_l$ são estimados por GMM.
No primeiro estágio estime por OLS o seguinte modelo:

$y_{jt} = \beta_0 + \beta_k k_{jt} + \beta_l l_{jt} + f^{−1}_t (k_{jt}, l_{jt},m_{jt}) + \varepsilon_{jt}$ (2)
$y_{jt} = \Phi_t(k_{jt}, l_{jt},m_{jt}) + \varepsilon_{jt}$

tal que $\omega_{jt} \equiv f^{−1}_t (k_{jt}, l_{jt},m_{jt})$, tal que m são os insumos intermediários. O termo $f^{−1}_t (k_{jt}, l_{jt},m_{jt})$ é um polinômio de quarta ordem. O polinômio é o seguinte:

$f^{−1}_t (k_{jt}, l_{jt},m_{jt}) = \sum_{i=0}^4 \beta_{k,l,i,4−i}k^i_{jt} l^{4−i}_{jt} + \sum_{i=0}^4 \beta_{k,m,i,4−i}k^i_{jt} m^{4−i}_{jt} + \sum_{i=0}^4 \beta_{m,l,i,4−i}m^i_{jt} l^{4−i}_{jt} $ (3)

Após estimar (2) calcule e salve o valor predito de $\Phi_t(·)$:

$ \hat{\Phi_t}(k_{jt}, l_{jt},m_{jt}) $(4)

Assumindo que a produtividade tem dinâmica AR(1), $\omega_{jt} = \rho \omega_{jt−1} + \xi_{jt}$, o
segundo estágio é formado pelos momentos
$E[(\xi_{jt} + \varepsilon_{jt}).Z_{jt}] = 0$

tal que $Z_{jt}$ é o vetor de instrumentos. Substituindo $\xi_{jt}$, $\varepsilon_{jt}$ e (4) na equação de momentos, temos:

$E[(y_{jt} - \beta_0 - \beta_k k_{jt} - \beta_l l_{jt} -\rho(\hat{\Phi_{t-1}}(k_{jt-1}, l_{jt-1},m_{jt-1}) - \beta_{0} - \beta_k k_{jt-1} - \beta_l l_{jt-1}))  \times \begin{pmatrix} 
  1 \\ 
  k_{jt} \\
  l_{jt-1} \\
  \hat{\Phi_{t-1}}(k_{jt-1}, l_{jt-1},m_{jt-1})
  \end{pmatrix}]$  (5)

O termo $\hat{\rho}$ pode ser estimado usando a estrutura AR(1) da produtividade $\omega_{jt} = \rho \omega_{jt−1} + \xi_{jt}$. Isto é, o momento necessário para estimar $\hat{\rho}$ não é incluido no sistema GMM. Portanto, para cada loop do algoritmo de estimação é possível estimar o $\hat{\rho}$ com base na produtividade.
Para reduzir o problema, estime $\beta_0$ no primeiro estágio, $\tilde{\beta_0}$, e substitua em (5). Alternativamente, se for conveniente assuma um valor fixo para $\hat{\rho}$, por
exemplo 0.9528. Dado isso, o modelo a ser estimado simplificado para

$E[(y_{jt} - \tilde{\beta_0} - \beta_k k_{jt} - \beta_l l_{jt} -\hat{\rho}(\hat{\Phi_{t-1}}(k_{jt-1}, l_{jt-1},m_{jt-1})- \tilde{beta_0} - \beta_k k_{jt-1} - \beta_l l_{jt-1}))  \times \begin{pmatrix} 
  k_{jt} \\
  l_{jt-1} \\
\end{pmatrix}]$ (6)

Observe que este modelo é exatamente identificado.


Base de dados

A base é de firmas chilenas de 1996 a 2006. As variáveis são:

• Y : ln do produto deflacionado, medido como valor adicionado

• sX : ln do estoque de capital deflacionado (k)

• fX1 : ln do emprego 1 (blue c)

• fX2 : ln do emprego 2 (white c)

• pX : ln de materiais (m)

• cX : ln do investimento deflacionado

• inv = cX

• idvar : id firma

• timevar : id tempo (ano)

A medida de emprego a ser usada deve ser a soma de fX1 e fX2.